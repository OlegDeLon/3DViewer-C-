\documentclass[12pt, a4paper]{article}

\usepackage[utf8]{inputenc}
\usepackage[T2A]{fontenc}
\usepackage[english,russian]{babel}
\usepackage{indentfirst}
\usepackage{graphicx}
\graphicspath{{../pictures/}}

\title{3DViewer v2.0}
\date{}

\begin{document}

\maketitle
\tableofcontents

\pagebreak

\section*{Описание}

Программа 3DViewer выполняет отображение 3D моделей, загруженных из файлов формата .obj, в каркасном виде.

Реализована на базе библиотеки Qt.

\section{Сборка}

Для того, чтобы запустить программу её необходимо собрать из исходников.
Сборка производится утилитой \textbf{make}.

Для сборки необходима утилита qmake 6.2.4

В папке с исходными файлами запустите утилиту \textbf{make} или \textbf{make all}.

Исполняемый файл \textbf{3DViewer} лежит в папке \textbf{build}.

\section{Тестирование}
Тестирование вычислительной части программы, отвечающей за афинные преобразования модели, производится с помощью библиотеки \textbf{check}.

Для тестирования необходимо вызвать \textbf{make} с целью \textbf{tests}.\
Покрытие тестами вызывается целью \textbf{gcov\_report}.

\section{Установка}
Тестировалось только на Ubuntu 22.04.
Установка вызывается целью \textbf{install}, копирует собранный исполняемый файл в папку \textbf{usr/bin/}. 

\section{Удаление}
Удаление производится вызовом цели \textbf{uninstall}, удаляя ранее скопированные файлы.
\vfill

\section{Основные возможности программы}

В программе 3DViewer возможно:
\begin{itemize}
    \item Загружать каркасную модель из файла формата .obj;
    \item Перемещать модель на заданное расстояние относительно осей X, Y, Z;
    \item Поворачивать модель за задданный угол вокруг своих осей X, Y, Z;
    \item Маштабировать модель за заданное значение.
\end{itemize}

\subsection{Настройки отображения}

Также в программе реализована возможность настройки отображения.

Имеется возможность выбора типа проекции (центральная, параллельная).

Настройка типа (сплошная, пунктирная), цвета, толщины линии.

Настройка типа (круглый, квадратный), размер и цвет вершин.

Настройка цвета фона.

Программа сохраняет настройки между перезапусками.

\subsection{Запись}

В программе реализована возможность записи полученных изоброжений в файл формата .bmp .jpeg.

Также имеется возможность записи "скринкастов" афинных преобразований в gif-анимацию (640х480, 10fps, 5s).

\end{document}
